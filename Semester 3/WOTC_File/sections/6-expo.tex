\chapter{Expository Essays}

\section{What is an Expository Essay?}
	The expository essay is a genre of essay that requires the student to investigate an idea, evaluate evidence, expound on the idea, and set forth an argument concerning that idea in a clear and concise manner. This can be accomplished through comparison and contrast, definition, example, the analysis of cause and effect, etc.

\section{Guidelines for writing Expository Essays}
	\begin{itemize}
		\item \textbf{A clear, concise, and defined thesis statement that occurs in the first paragraph of the essay.}
		It is essential that this thesis statement be appropriately narrowed to follow the guidelines set forth in the assignment. If the student does not master this portion of the essay, it will be quite difficult to compose an effective or persuasive essay.
		
		\item \textbf{Clear and logical transitions between the introduction, body, and conclusion.}
		Transitions are the mortar that holds the foundation of the essay together. Without logical progression of thought, the reader is unable to follow the essay’s argument, and the structure will collapse.
		
		\item \textbf{Body paragraphs that include evidential support.}
		Each paragraph should be limited to the exposition of one general idea. This will allow for clarity and direction throughout the essay. What is more, such conciseness creates an ease of readability for one’s audience. It is important to note that each paragraph in the body of the essay must have some logical connection to the thesis statement in the opening paragraph.
		
		\item \textbf{Evidential support (whether factual, logical, statistical, or anecdotal).}
		Often times, students are required to write expository essays with little or no preparation; therefore, such essays do not typically allow for a great deal of statistical or factual evidence.
		
		\item \textbf{A bit of creativity!}
		Though creativity and artfulness are not always associated with essay writing, it is an art form nonetheless. Try not to get stuck on the formulaic nature of expository writing at the expense of writing something interesting. Remember, though you may not be crafting the next great novel, you are attempting to leave a lasting impression on the people evaluating your essay.
		
		\item \textbf{A conclusion that does not simply restate the thesis, but readdresses it in light of the evidence provided.}
		It is at this point of the essay that students will inevitably begin to struggle. This is the portion of the essay that will leave the most immediate impression on the mind of the reader. Therefore, it must be effective and logical. Do not introduce any new information into the conclusion; rather, synthesize and come to a conclusion concerning the information presented in the body of the essay.
	\end{itemize}

\section{Examples}

\begin{mytcbox}{Beyond Identity}
A student’s life is often hectic. Moving from class to class, ingesting lots of information, a load
of coursework and preparing for examinations is a lot to handle. The leisure time a student gets
should be treasured and used wisely. Sadly, most students in today’s society spend their free
time indulging in activities that are harmful to their well-being such as ingesting alcohol and
drugs. Precious leisure time can be used to decompress using meaningful but still relaxing
activities. Leisure time should contribute to a student’s physical, mental and spiritual wellbeing. These three areas will contribute to a more wholesome student.
\\
\\
\textbf{Exercise and Sport}
\\
Given that most classes are sedentary activities, a student should spend their time get their
bodies active through exercise and other physical activities. Leisure time can be used as a way
to look after your health. The body’s well-being undoubtedly constitutes the physical aspect.
When a student is in better physical shape, their concentration, energy levels and participation
in class also increase. Most students sit while in class. Medical research shows that prolonged
sessions of unadulterated sitting have adverse effects on the body’s health by exercising during
their leisure time, students can counteract these negative consequences. Exercise can include
endurance activities such as running, swimming, martial arts and bike riding. It could also
include power exercises such as weightlifting. Sports are also an excellent choice in this regard.
You get to work your body out while having fun at the same time.
\\
\\
\textbf{Artistic Pursuits}
\\
Students should be involved in arts during their leisure time. This activity is vital especially for
those students studying scientific courses. Those studying artistic courses should practice other
arts as well. Arts are critical to developing our creativity. Creativity assists students to be more
critical and original thinkers in their day to day lives. Studying new skills causes the brain to
grow. It is challenging as well as exciting. Arts are also a way of self-expression. Self-expression
is vital in giving a student a release from the pressures of everyday life. A student may also
discover hidden talents in this regard which he may go on to make a living out later in life. The
arts could teach a student how to live passionately which is solely lacking in the modern world.
A student gains much virtue from drawing, painting, and writing among other arts. Such virtues
spill over into other areas of their lives.
\\
\\
\textbf{Relaxation}
\\
Relaxation brings about the tranquility that a student cannot find anywhere else. In modern
society’s hurried ways, to slow down even for a few minutes each day will bring peace to a
student’s life. It helps to achieve peace of mind. A student can calm down and see what is
crucial in their lives. Every endeavor is carried out with more clarity. An undercurrent of peace
is very healthy while carrying on routine activities in a student’s day to day life. Meditation is a
practice that would help a lot of students in schools currently suffering and in pain.
The activities outlined above seek to make a student more balanced. Since schoolwork is more
specific and mainly deals with the intellect, students should find activities that are not
curriculum oriented. Activities that make them human beings that are closer to their nature.
Activities that give them joy and bring them greater understanding not only of the world but
also of themselves.
Also, activities that help them to exercise their brains and relax. After all work and no play
makes Jack a dull boy and school is the last place anyone wants to feel dull and detached.	
\end{mytcbox}