\chapter{Basics about Essay Writing}

\section{What are Essays?}
An essay is a "\textbf{short formal piece of writing..dealing with a single subject}". It is typically written to try to persuade the reader using selected research evidence.

An essay is a piece of non-fiction writing with a clear structure: an introduction, paragraphs with evidence and a conclusion. Writing an essay is an important skill in English and allows you to show your knowledge and understanding of the texts you read and study.

It is important to plan your essay before you start writing so that you write clearly and thoughtfully about the essay topic. Evidence, in the form of quotations and examples, is the foundation of an effective essay and provides proof for your points.

\section{Why do we write essays?}
The purpose of an essay is to show your understanding, views or opinions in response to an essay question, and to persuade the reader that what you are writing makes sense and can be backed up with evidence. In a literature essay, this usually means looking closely at a text (for example, a novel, poem or play) and responding to it with your ideas.

Essays can focus on a particular section of a text, for example, a particular chapter or scene, or ask a big picture question to make you think deeply about a character, idea or theme throughout the whole text.

Often essays are questions, for example, ‘How does the character Jonas change in the novel, The Giver by Lois Lowry?’ or they can be written using command words to tell you what to do, for example ‘Examine how the character Jonas changes in the novel, The Giver by Lois Lowry.’

It is important to look carefully at the essay question or title so that you keep your essay focused and relevant. If the essay tells you to compare two specific poems, you shouldn’t just talk about the two poems separately and you shouldn’t bring in lots of other poems.

\section{Types of Essays}
	\subsection{Argumentative or Persuasive Essay}
		An argumentative or persuasive essay takes a strong position on a topic through the use of supporting evidence.
		
		It:
		\begin{itemize}
			\item Requires thorough research and investigation of the topic.
			\item Includes a clear, strong thesis statement that is debatable.
			\item Considers and refutes alternative arguments with cited evidence, statistics, and facts.
			\item Uses fair, objective language with a well-rounded understanding of the topic.
		\end{itemize}
	
	\subsection{Comparative Essays}
		A comparative essay requires comparison and/or contrast of at least two or more items.
		
		It:
		\begin{itemize}
			\item Attempts to build new connections or note new similarities or differences about the topic(s).
			\item Typically focuses on items of the same class, i.e. two political systems (i.e. democracy or communism) or two theories (i.e. behaviorism versus constructivism).
		\end{itemize}
	
	\subsection{Expository Essays}
		The purpose of an expository essay is to describe or explain a specific topic.
		
		It:
		\begin{itemize}
			\item Uses factual information.
			\item Is written from the third-person point of view.
			\item Does not require a strong, formal argument.
		\end{itemize}

	\subsection{Narrative Essays}
		A narrative essay tells a story or describes an event in order to illustrate a key point or idea.
		
		It:
		\begin{itemize}
			\item Uses descriptive and sensory information to communicate to the reader.
			\item Are often subjective rather than objective.
			\item Usually written from the first-person or third-person point of view.
			\item May be entertaining or informative.
		\end{itemize}

\section{Planning an Essay}
	\subsubsection{It is important to plan before you start writing an essay.}
		The essay question or title should provide a clear focus for your plan. Exploring this will help you make decisions about what points are relevant to the essay.
	
	\subsubsection{What are you being asked to consider?}
		Organise your thoughts. Researching, mind mapping and making notes will help sort and prioritise your ideas. If you are writing a literature essay, planning will help you decide which parts of the text to focus on and what points to make.
		
	\subsubsection{There are three main parts to an essay:}
		\begin{enumerate}
			\item Introduction
				\begin{itemize}
					\item An introduction should focus directly on the essay question or title and aim to present your main idea in your answer. It briefly introduces your main ideas and arguments.
				\end{itemize}
			\item Main body, divided into paragraphs
				\begin{itemize}
					\item This is where you take your ideas and explore them in detail in separate paragraphs. You might want to start each paragraph with a topic sentence that summarises the main idea of the paragraph before bringing in your evidence and examples. A topic sentence acts like a mini introduction to the paragraph.
				\end{itemize}
			\item Conclusion
				\begin{itemize}
					\item A conclusion is the final paragraph of your essay. It should tie all the loose ends of your argument together.
				\end{itemize}
		\end{enumerate}
	
\section{Answering the question}
	When writing an essay it is important to \textbf{answer the question} and not just write everything you know about a particular subject. Part of the secret to writing a good essay is to carefully choose what is interesting and relevant.
	
	To make sure you answer the question, the first step is to be clear: what does the essay want you to write about? In other words, what are the key words or phrases in the essay question or title?
	
	The second step in answering the question is then to think about everything you do know about the topic and decide which ideas are the strongest and most interesting to write about.
	
	If the essay asks you about one of the characters that must be your focus. For example, if you are asked to write about a character, like \textit{William Shakespeare}, then it is not a good idea to spend paragraphs describing other characters – however important they might be to the story.

\section{Using evidence}
	Evidence is the foundation of an effective essay and provides proof for your points.
	
	For an essay about a piece of literature, the best evidence will come from the text itself.
	
	Back up each of your supporting statements with evidence. The evidence should be relevant and clearly connected to the point you’re making.
	
	In a literary essay, evidence could take the form of:
	\begin{itemize}
		\item Quotations from the text, for example, if the essay focus was on the character of The Giver, it would be useful to explain that Lowry first describes The Giver as ‘tired’ ‘old’ and ‘weighted’ which suggests that he is suffering.
		\item Examples that describe the text. For example, for the same essay focusing on the character of The Giver, it might be interesting to explore the way that The Giver chooses the memories for Jonas to experience or his sadness at the loss of Rosemary. You can't quote all of this so you have to summarise the text.
	\end{itemize}

\section{Referring to literary devices}
	In literature essays, you are often asked to look closely at how the writer writes and analyse the language used. For example, what words and phrases does the writer use? Do they use any literary devices like metaphors or similes and what is the effect of them? Do they repeat words or create other patterns with language? It is worth looking carefully at quotations to notice what the writer is doing and why they might be doing it.
	
	For example in The Giver, Jonas is given access to lots of memories – some wonderful, some painful. At one point, he receives a memory of a ‘bright, breezy day on a clear turquoise lake, and above him the white sail of the boat billowing as he moved along in the brisk wind.’ In an essay you could use this as an example of a positive memory that has been lost to everyone except Jonas and The Giver. You could then zoom in deeper on the quotation and comment on the language and its effect. You might notice the alliteration of the ‘b’ sound or the use of colour imagery to help to create such a positive memory.
	
	In a literature essay it is useful to know how to use technical terms such as metaphor, simile or imagery. It is also useful to use technical vocabulary for writing about literature such as plot, character, setting or theme.