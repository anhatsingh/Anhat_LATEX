\chapter{Argumentative Essays}

\section{What are Argumentative Essays?}
	The argumentative essay is a genre of writing that requires the student to investigate a topic; collect, generate, and evaluate evidence; and establish a position on the topic in a concise manner.
	
	Argumentative essay assignments generally call for extensive research of literature or previously published material. Argumentative assignments may also require empirical research where the student collects data through interviews, surveys, observations, or experiments. Detailed research allows the student to learn about the topic and to understand different points of view regarding the topic so that she/he may choose a position and support it with the evidence collected during research. Regardless of the amount or type of research involved, argumentative essays must establish a clear thesis and follow sound reasoning.
	
\section{Structure of Argumentative Essays}
	\subsubsection{A clear, concise, and defined thesis statement that occurs in the first paragraph of the essay.}
		In the first paragraph of an argument essay, students should set the context by reviewing the topic in a general way. Next the author should explain why the topic is important (exigence) or why readers should care about the issue. Lastly, students should present the thesis statement. It is essential that this thesis statement be appropriately narrowed to follow the guidelines set forth in the assignment. If the student does not master this portion of the essay, it will be quite difficult to compose an effective or persuasive essay.
	
	\subsubsection{Clear and logical transitions between the introduction, body, and conclusion.}
		Transitions are the mortar that holds the foundation of the essay together. Without logical progression of thought, the reader is unable to follow the essay’s argument, and the structure will collapse. Transitions should wrap up the idea from the previous section and introduce the idea that is to follow in the next section.
	
	\subsubsection{Body paragraphs that include evidential support.}
		Each paragraph should be limited to the discussion of one general idea. This will allow for clarity and direction throughout the essay. In addition, such conciseness creates an ease of readability for one’s audience. It is important to note that each paragraph in the body of the essay must have some logical connection to the thesis statement in the opening paragraph. Some paragraphs will directly support the thesis statement with evidence collected during research. It is also important to explain how and why the evidence supports the thesis (warrant).
		
		However, argumentative essays should also consider and explain differing points of view regarding the topic. Depending on the length of the assignment, students should dedicate one or two paragraphs of an argumentative essay to discussing conflicting opinions on the topic. Rather than explaining how these differing opinions are wrong outright, students should note how opinions that do not align with their thesis might not be well informed or how they might be out of date.
	
	\subsubsection{Evidential support (whether factual, logical, statistical, or anecdotal).}
		The argumentative essay requires well-researched, accurate, detailed, and current information to support the thesis statement and consider other points of view. Some factual, logical, statistical, or anecdotal evidence should support the thesis. However, students must consider multiple points of view when collecting evidence. As noted in the paragraph above, a successful and well-rounded argumentative essay will also discuss opinions not aligning with the thesis. It is unethical to exclude evidence that may not support the thesis. It is not the student’s job to point out how other positions are wrong outright, but rather to explain how other positions may not be well informed or up to date on the topic.
	
	\subsubsection{A conclusion that does not simply restate the thesis, but readdresses it in light of the evidence provided.}
		It is at this point of the essay that students may begin to struggle. This is the portion of the essay that will leave the most immediate impression on the mind of the reader. Therefore, it must be effective and logical. Do not introduce any new information into the conclusion; rather, synthesize the information presented in the body of the essay. Restate why the topic is important, review the main points, and review your thesis. You may also want to include a short discussion of more research that should be completed in light of your work.
	
\section{A COMPLETE ARGUEMENT}
	Perhaps it is helpful to think of an essay in terms of a conversation or debate with a classmate. If I were to discuss the cause of World War II and its current effect on those who lived through the tumultuous time, there would be a beginning, middle, and end to the conversation. In fact, if I were to end the argument in the middle of my second point, questions would arise concerning the current effects on those who lived through the conflict. Therefore, the argumentative essay must be complete, and logically so, leaving no doubt as to its intent or argument.

\section{THE FIVE-PARAGRAPH ESSAY}
	A common method for writing an argumentative essay is the five-paragraph approach. This is, however, by no means the only formula for writing such essays. If it sounds straightforward, that is because it is; in fact, the method consists of (a) an introductory paragraph (b) three evidentiary body paragraphs that may include discussion of opposing views and (c) a conclusion.

\section{LONGER ARGUMENTATIVE ESSAYS}
	Complex issues and detailed research call for complex and detailed essays. Argumentative essays discussing a number of research sources or empirical research will most certainly be longer than five paragraphs. Authors may have to discuss the context surrounding the topic, sources of information and their credibility, as well as a number of different opinions on the issue before concluding the essay. Many of these factors will be determined by the assignment.

\section{Examples}
\begin{mytcbox}{Malaria}
	Malaria is an infectious disease caused by parasites that are transmitted to people through female Anopheles mosquitoes. Each year, over half a billion people will become infected with malaria, with roughly 80\% of them living in Sub-Saharan Africa. Nearly half a million people die of malaria every year, most of them young children under the age of five. Unlike many other infectious diseases, the death toll for malaria is rising. While there have been many programs designed to improve access to malaria treatment, the best way to reduce the impact of malaria in Sub-Saharan Africa is to focus on reducing the number of people who contract the disease in the first place, rather than waiting to treat the disease after the person is already infected.
	\\
	\\
	There are multiple drugs available to treat malaria, and many of them work well and save lives, but malaria eradication programs that focus too much on them and not enough on prevention haven’t seen long-term success in Sub-Saharan Africa. A major program to combat malaria was WHO’s Global Malaria Eradication Programme. Started in 1955, it had a goal of eliminating malaria in Africa within the next ten years. Based upon previously successful programs in Brazil and the United States, the program focused mainly on vector control. This included widely distributing chloroquine and spraying large amounts of DDT. More than one billion dollars was spent trying to abolish malaria. However, the program suffered from many problems and in 1969, WHO was forced to admit that the program had not succeeded in eradicating malaria. The number of people in Sub-Saharan Africa who contracted malaria as well as the number of malaria deaths had actually increased over 10\% during the time the program was active.
	\\
	\\	
	One of the major reasons for the failure of the project was that it set uniform strategies and policies. By failing to consider variations between governments, geography, and infrastructure, the program was not nearly as successful as it could have been. Sub-Saharan Africa has neither the money nor the infrastructure to support such an elaborate program, and it couldn’t be run the way it was meant to. Most African countries don't have the resources to send all their people to doctors and get shots, nor can they afford to clear wetlands or other malaria prone areas. The continent’s spending per person for eradicating malaria was just a quarter of what Brazil spent. Sub-Saharan Africa simply can’t rely on a plan that requires more money, infrastructure, and expertise than they have to spare.
	\\
	\\	
	Additionally, the widespread use of chloroquine has created drug resistant parasites which are now plaguing Sub-Saharan Africa. Because chloroquine was used widely but inconsistently, mosquitoes developed resistance, and chloroquine is now nearly completely ineffective in Sub-Saharan Africa, with over 95\% of mosquitoes resistant to it. As a result, newer, more expensive drugs need to be used to prevent and treat malaria, which further drives up the cost of malaria treatment for a region that can ill afford it.
	\\
	\\
	Instead of developing plans to treat malaria after the infection has incurred, programs should focus on preventing infection from occurring in the first place. Not only is this plan cheaper and more effective, reducing the number of people who contract malaria also reduces loss of work/school days which can further bring down the productivity of the region.
	\\
	\\
	Reducing the number of people who contract malaria would also reduce poverty levels in Africa significantly, thus improving other aspects of society like education levels and the economy. Vector control is more effective than treatment strategies because it means fewer people are getting sick. When fewer people get sick, the working population is stronger as a whole because people are not put out of work from malaria, nor are they caring for sick relatives. Malaria-afflicted families can typically only harvest 40\% of the crops that healthy families can harvest. Additionally, a family with members who have malaria spends roughly a quarter of its income treatment, not including the loss of work they also must deal with due to the illness. It’s estimated that malaria costs Africa 12 billion USD in lost income every year. A strong working population creates a stronger economy, which Sub-Saharan Africa is in desperate need of.	
\end{mytcbox}