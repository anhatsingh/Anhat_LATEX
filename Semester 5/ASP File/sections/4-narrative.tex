\chapter{Narrative Essays}

\section{What is a Narrative Essay?}
	When writing a narrative essay, one might think of it as telling a story. These essays are often anecdotal, experiential, and personal—allowing students to express themselves in a creative and, quite often, moving ways. 
	
	When writing a narrative essay, remember that you are sharing sensory and emotional details with the reader.
	
	Narrative essays tell a vivid story, usually from one person's viewpoint. A narrative essay uses all the story elements — a beginning, middle and ending, as well as plot, characters, setting and climax — bringing them together to complete the story. The focus of a narrative essay is the plot, which is told with enough detail to build to a climax.

\section{Guidelines for writing a Narrative Essay}
	\begin{itemize}
		\item \textbf{If written as a story, the essay should include all the parts of a story.} This means that you must include an introduction, plot, characters, setting, climax, and conclusion.
		\item \textbf{When would a narrative essay not be written as a story?} A good example of this is when an instructor asks a student to write a book report. Obviously, this would not necessarily follow the pattern of a story and would focus on providing an informative narrative for the reader.
		\item \textbf{The essay should have a purpose.} Make a point! Think of this as the thesis of your story. If there is no point to what you are narrating, why narrate it at all?
		\item \textbf{The essay should be written from a clear point of view.} It is quite common for narrative essays to be written from the standpoint of the author; however, this is not the sole perspective to be considered. Creativity in narrative essays oftentimes manifests itself in the form of authorial perspective.
		\item \textbf{Use clear and concise language throughout the essay.} Much like the descriptive essay, narrative essays are effective when the language is carefully, particularly, and artfully chosen. Use specific language to evoke specific emotions and senses in the reader.
		\item \textbf{The use of the first person pronoun ‘I’ is welcomed.} Do not abuse this guideline! Though it is welcomed it is not necessary—nor should it be overused for lack of clearer diction.
		\item \textbf{As always, be organized!} Have a clear introduction that sets the tone for the remainder of the essay. Do not leave the reader guessing about the purpose of your narrative. Remember, you are in control of the essay, so guide it where you desire (just make sure your audience can follow your lead).
	\end{itemize}

\section{Examples}
	\begin{mytcbox}{A Teeny, Tiny Treasure Box}
		She took me by the hand and walked me into the lobby like a five-year old child. Didn’t she know I was pushing 15? This was the third home Nancy was placing me in - in a span of eight months. I guess she felt a little sorry for me. The bright fluorescent lights threatened to burn my skin as I walked towards a bouncy-looking lady with curly hair and a sweetly-smiling man. They called themselves Allie and Alex. Cute, I thought.
		\\
		\\		
		After they exchanged the usual reams of paperwork, it was off in their Chevy Suburban to get situated into another new home. This time, there were no other foster children and no other biological children. Anything could happen.
		\\
		\\
		Over the next few weeks, Allie, Alex, and I fell into quite a nice routine. She’d make pancakes for breakfast, or he’d fry up some sausage and eggs. They sang a lot, even danced as they cooked. They must have just bought the house because, most weekends, we were painting a living room butter yellow or staining a coffee table mocha brown.
		\\
		\\
		I kept waiting for the other shoe to drop. When would they start threatening a loss of pancakes if I didn’t mow the lawn? When would the sausage and eggs be replaced with unidentifiable slosh because he didn’t feel like cooking in the morning? But, it never happened. They kept cooking, singing, and dancing like a couple of happy fools.
		\\
		\\
		It was a Saturday afternoon when Allie decided it was time to paint the brick fireplace white. As we crawled closer to the dirty old firepit, we pulled out the petrified wood and noticed a teeny, tiny treasure box. We looked at each other in wonder and excitement. She actually said, “I wonder if the leprechauns left it!” While judging her for being such a silly woman, I couldn’t help but laugh and lean into her a little.
		\\
		\\
		Together, we reached for the box and pulled it out. Inside was a shimmering solitaire ring. Folded underneath was a short piece of paper that read:
		\\
		\\
		“My darling, my heart. Only 80 days have passed since I first held your hand. I simply cannot imagine my next 80 years without you in them. Will you take this ring, take my heart, and build a life with me? This tiny little solitaire is my offering to you. Will you be my bride?”
		\\
		\\
		As I stared up at Allie, she asked me a question. “Do you know what today is?” I shook my head. “It’s May 20th. That’s 80 days since Nancy passed your hand into mine and we took you home.”
		\\
		It turns out, love comes in all shapes and sizes, even a teeny, tiny treasure box from a wonderfully silly lady who believes in leprechauns.
	\end{mytcbox}