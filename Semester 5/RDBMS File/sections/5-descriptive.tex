\chapter{Descriptive Essays}

\section{What is a descriptive Essay?}
	The descriptive essay is a genre of essay that asks the student to describe something—object, person, place, experience, emotion, situation, etc. This genre encourages the student’s ability to create a written account of a particular experience. What is more, this genre allows for a great deal of artistic freedom (the goal of which is to paint an image that is vivid and moving in the mind of the reader)
	
	One might benefit from keeping in mind this simple maxim: If the reader is unable to clearly form an impression of the thing that you are describing, try, try again!

\section{Guidelines for writing a descriptive essay}
	\begin{itemize}
		\item \textbf{Take time to brainstorm.} If your instructor asks you to describe your favorite food, make sure that you jot down some ideas before you begin describing it. For instance, if you choose pizza, you might start by writing down a few words: sauce, cheese, crust, pepperoni, sausage, spices, hot, melted, etc. Once you have written down some words, you can begin by compiling descriptive lists for each one.
		
		\item \textbf{Use clear and concise language.} This means that words are chosen carefully, particularly for their relevancy in relation to that which you are intending to describe.
		
		\item \textbf{Choose vivid language.} Why use horse when you can choose stallion? Why not use tempestuous instead of violent? Or why not miserly in place of cheap? Such choices form a firmer image in the mind of the reader and often times offer nuanced meanings that serve better one’s purpose.
		
		\item \textbf{Use your senses!} Remember, if you are describing something, you need to be appealing to the senses of the reader. Explain how the thing smelled, felt, sounded, tasted, or looked. Embellish the moment with senses.
		
		\item \textbf{What were you thinking?!} If you can describe emotions or feelings related to your topic, you will connect with the reader on a deeper level. Many have felt crushing loss in their lives, or ecstatic joy, or mild complacency. Tap into this emotional reservoir in order to achieve your full descriptive potential.
		
		\item \textbf{Leave the reader with a clear impression.} One of your goals is to evoke a strong sense of familiarity and appreciation in the reader. If your reader can walk away from the essay craving the very pizza you just described, you are on your way to writing effective descriptive essays.
		
		\item \textbf{Be organized!} It is easy to fall into an incoherent rambling of emotions and senses when writing a descriptive essay. However, you must strive to present an organized and logical description if the reader is to come away from the essay with a cogent sense of what it is you are attempting to describe.
	\end{itemize}

\section{Examples}
\begin{mytcbox}{The Thunderstorm}
	I watched a thunderstorm, far out over the sea. It began quietly, and with nothing visible except tall dark clouds and a rolling tide. There was just a soft murmur of thunder as I watched the horizon from my balcony. Over the next few minutes, the clouds closed and reflected lightning set the rippling ocean aglow. The thunderheads had covered up the sun, shadowing the vista. It was peaceful for a long time.
	\\
	\\
	I was looking up when the first clear thunderbolt struck. It blazed against the sky and sea; I could see its shape in perfect reverse colors when I blinked. More followed. The thunder rumbled and stuttered as if it could hardly keep up. There were openings in the cloud now, as if the sky were torn, and spots of brilliant blue shone above the shadowed sea.
	\\
	\\
	I looked down then, watching the waves. Every bolt was answered by a moment of spreading light on the surface. The waves were getting rough, rising high and crashing hard enough that I could hear them.
	\\
	\\
	Then came the rain. It came all at once and in sheets, soaking the sand, filling the sea. It was so dense I could only see the lightning as flashes of light. It came down so hard the thunder was drowned. Everything was rhythmic light and shadow, noise and silence, blending into a single experience of all five senses.
	\\
	\\
	In an instant it stopped. The storm broke. The clouds came apart like curtains. The rain still fell, but softly now. It was as if there had never been a storm at all, except for a single signature. A rainbow, almost violently bright, spread above and across the water. I could see the horizon again.
\end{mytcbox}

\begin{mytcbox}{The Mantis}
	The orchid mantis, is a remarkable creature. Against any opponent but a careful entomologist with a cardboard box, the mantis is a lethal hunter and master of camouflage. Its four front legs, head and thorax are covered in delicate structures resembling colorful flower petals. In appearance, it looks like nothing so much as a praying mantis covered in beautiful painted fans.
	\\
	\\
	As for its behavior, like any good mantis, it is an ambush predator. It takes full advantage of its unique appearance, settling amongst the petals of orchids and awaiting visiting insects. It favors butterflies and moths for its meals, but will happily take any insect on offer. Indeed, it need not even be an insect: particularly voracious orchid mantises have been known to feed on small lizards, frogs, mice and even birds.
	\\
	\\
	Its behavior among its own kind is no different. Like many mantises, orchid mantises are opportunistic cannibals. They don't go out of their way to devour their own kind, but should one stray into striking range of another when it's feeling peckish, it may well become a meal. H. coronatus is not recorded as performing the praying mantis's infamous reproductive cannibalism, however.
	\\
	\\
	Its relationship to humans is neutral, verging on positive. H. coronatus is not an ally of the committed gardener like the aphid-devouring ladybug, but it will nibble on any pests that present themselves. Aside from that, the orchid mantis is only valuable to humans for its extraordinary beauty.
	\\
	\\
	Hymenopus coronatus is an example of a unique form of beauty that exists only in nature, careless of human judgment, designed for function rather than form, but still capable of making an observer catch their breath at its strange loveliness.
\end{mytcbox}